\title{Toy Problem: MOCU for Model-Inadequacy}

\author{}
\date{}

\documentclass[11pt]{article}
\usepackage[margin=1in]{geometry}
\usepackage{amsmath, amsfonts, bm, bbm, graphicx, caption, subcaption}
\usepackage{titling}
\usepackage{algorithm}
\usepackage{algpseudocode}

\setlength{\droptitle}{-0.75in}   % Adjust title margin
\pagenumbering{gobble}

\renewcommand\refname{Appendix 3: Bibliography \& References Cited}

\begin{document}
\newcommand*{\vertbar}{\rule[-1ex]{0.5pt}{2.5ex}}
\newcommand*{\horzbar}{\rule[.5ex]{2.5ex}{0.5pt}}
\maketitle

\newpage
\pagenumbering{arabic}

\section{What is the point of this example?}

This is supposed to be a toy problem demonstrating how MOCU might be
applied to inferring/tuning a model for a full-physics process. The
motivating setting is materials discovery. We assume the following:

\begin{enumerate}
\item Doing an experiment (i.e., fabricating a nanoscale material at CFN) is expensive
\item There are several types of experiments we could do. As a concrete example, if iron is the host material, we could choose to dope with carbon (traditional steel), or we could dope with aluminum, or nickel. Each of these choices would result in a different material system with a different phase diagram and different properties. In the language of MOCU, we denote this set of possible experimental designs as $\Psi = \lbrace \psi_1 \dots \psi_p \rbrace$. If our choices were Fe-C, Fe-Al, and Fe-Ni, then $p=3$.
\item We have a computational surrogate model for the real experimental process, but this surrogate is an imperfect description of that process, because we don't know a full description of the physics.
\item The scientific uncertainty shows up in the surrogate model as a parameterized set of RHS terms. We have some prior distribution over these parameters. In MOCU terms, this is the uncertainty class, and it is parameterized as $\Theta = \lbrace \theta_1 \dots \theta_n \rbrace$, with $\theta_i \sim \rho(\theta_i)$ being the prior distribution over one of the parameters. Because each experimental design $\psi_j$ represents a unique physical system, there will be a unique set of parameter values $\theta_i$ that produce a ``best agreement'' with data drawn from the real system for $\psi_j$, and these values will be different for a different $\psi$.
\item We would like to use experiments to reduce the uncertainty in our parameterized surrogate models
\end{enumerate}

\section{Description in terms of MOCU terminology}

\subsection{Experimental system}

An ``experiment'' in this toy example consists of a solution to the following two-state ODE:

\begin{equation}
\frac{d}{dt}
\begin{bmatrix}
x_1 \\
x_2
\end{bmatrix}
=
\begin{bmatrix}
\lambda_1 (x_1 - 1) \\
\lambda_2 \left(x_2 - \mathcal{C}(x_1,x_2)\right)(x_2-1)x_2
\end{bmatrix}
\label{eq:fullphysics}
\end{equation}

All parameters are known perfectly, so this system is completely
deterministic (with specification of initial conditions on
$\bm{x}$). We assume $\lambda_1=-0.01$, $\lambda_2 = -1$, and
$\mathcal{C}(x_1,x_2) = \mathcal{\bar{C}} + 0.1 \; \text{sin}(2 \pi k
x_1)$ with $k=5$. $\mathcal{\bar{C}}$ is a material-specific parameter
that will depend on our choice of experimental design (our choice of
$\psi$). In our problem, we assume there are three experimental
choices: $\Psi = \lbrace \psi_1 , \psi_2 , \psi_3 \rbrace$, and that
$\mathcal{\bar{C}}(\Psi) = \lbrace 0.4 , 0.5 ,
0.6 \rbrace$. Intuitively, we can design experiments on one of three
possible material systems $\psi_j$, and the material properties (phase
boundary location) depends on the choice of our material system
through $\mathcal{\bar{C}}(\psi_j)$.

It can be seen from the structure of the problem that $x_2$ will
asymptote to either 0 or 1, depending on the initial conditions. This
mimics a two-phase material: the final phase would be represented by
the final value that $x_2$ asymptotes to. The ``phase diagram'' for
this material is shown in Fig.~\ref{fig:experiments}, along with a few
trajectories.

\begin{figure}[htb]
  \centering
  \includegraphics[width=0.7\textwidth]{experiments.png}
  \caption{Experimental ``phase diagram''. Because $|\lambda_2| \gg |\lambda_1|$, the phase boundary is almost equal to $\mathcal{C}(x_1,x_2)$, which is shown in dashed-blue.}
  \label{fig:experiments}
\end{figure}

\subsection{Surrogate model}

Evaluating the experimental system is ``hard'', so we have a cheap
surrogate that we may query in its place. However, assume we don't
know the full physical process (Eq.~\ref{eq:fullphysics}), so our
model is incomplete. Our model will look like this:

\begin{equation}
dx_2 = \lambda_2 \left(x_2 - \psi \right)(x_2-1)x_2 + \theta dW
\label{eq:model}
\end{equation}

This is a stochastic differential equation, and $dW$ is a Wiener
process with zero mean and unit variance. Here, $\mathcal{\bar{C}} =
0.5$ and $\theta$ is an uncertain parameter that is supposed to model
the unknown physics. We assume a prior $\theta \sim \rho(\theta)$.

\begin{figure}[htb]
  \centering
  \includegraphics[width=0.9\textwidth]{model_theta0p03.png}
  \caption{{\it Left:} experimental system, showing the evolution of $x_2(t)$. {\it Right:} Surrogate evolution with $\theta = 0.03$. Colorscale is according to the initial condition value.}
  \label{fig:fullandmodel}
\end{figure}

\subsection{OED Goal}

Our goal is to locate materials that produce phase 1 with high
probability, across all $\psi$ and uncertainty $\theta$. An experiment
will consist of specifying an initial condition pair $(x_1,x_2)(t=0)$
and solving Eq.~\ref{eq:fullphysics}. The cost function we use for
MOCU is:

\begin{equation}
C(\theta , \psi) = | x_2(t=t_f , \theta) - 1 |
\end{equation}

The intuitive explanation for what we are doing as follows. As shown
in Fig.~\ref{fig:experiments} and \ref{fig:fullandmodel}, the mapping
from $x_2(t=0)$ to $x_2(t=t_f)$ is most uncertain around
$x_2(t=0)=0.5$. This is where the surrogate is least self-consistent:
a given initial condition may asymptote to 0 or 1, with high
unpredictability. We would like to avoid sampling in this region where
the model is ``unreliable'', and instead sample from those regions
where the model has high confidence across all values of $\theta$
(which would correspond to $x_2(t=0)$ values farther from 0.5). In the
analogy of materials, we are fitting the surrogate model to match data
from the regions of the phase diagram that are farther from the
boundary and involve less unpredictability in final phase.

\subsection{Results}



%% \begin{figure}[htb]
%%     \centering
%%     \begin{subfigure}[t]{0.4\textwidth}
%%         \centering
%%         \includegraphics[width=0.95\textwidth]{figures/energy_spectrum.png}
%%         \caption{Energy spectrum and scale discretization.}
%%     \end{subfigure}
%%     ~ 
%%     \begin{subfigure}[t]{0.57\textwidth}
%%         \centering
%%         \includegraphics[width=0.95\textwidth]{figures/energy_cascade.png}
%%         \caption{Scales resolved for each linking term.}
%%     \end{subfigure}
%%     \caption{Schematic depicting how the proposed methodology applies to a hypothetical multiscale system with dense scale content. In my method, I first divide the problem into $N$ scales then learn individual ROMs and local-scale linking terms using $N$ simulations that contain, at most, 3 scales each.}
%%   \label{fig:cascade}
%% \end{figure}


\end{document}
